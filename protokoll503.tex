\documentclass[11pt,ngerman,a4paper]{article}
%Gummi|061|=)
\usepackage{amsmath}
\usepackage{a4wide}
\usepackage{url}
\usepackage{amsthm}
\usepackage{amsbsy}
\usepackage{amssymb}
\usepackage[utf8]{inputenc}
\usepackage{rotating} 
\usepackage{here}
\usepackage{graphicx}
\usepackage{paralist}
\usepackage{selinput}
\usepackage[separate-uncertainty=true]{siunitx}
\usepackage{booktabs}
\sisetup{}
\SelectInputMappings{%
adieresis={ä},
germandbls={ß},
}
\title{\textbf{Versuch V503: Milikan-Versuch}}
\author{Martin Bieker\\
		Julian Surmann\\
		\\
		Durchgef\"{u}hrt am 03.06.2014\\
		TU Dortmund}
\date{}
\usepackage{graphicx}
\begin{document}
\renewcommand\tablename{Tabelle}
\renewcommand\figurename{Abbildung}
\maketitle
\thispagestyle{empty}
\newpage
\clearpage
\setcounter{page}{1}


\section{Einleitung}
In diesem Versuch soll die Elementarladung $e$ mit Hilfe der \"Oltr\"opfchenmethode nach Milikan bestimmt werden.
\section{Theorie}
Idee des hier beschrieben Versuchs ist die Bestimmung der Elementarladung durch Untersuchungen des Verhaltens von \"Oltropfen mit geringer Ladung im elektrischen Feld eines Kondensators. 
\subsection{Gravitation und Stokes-Reibung}
Zun\"achst werden die auf einen Tropfen wirkenden Kr\"afte bei abgeschalteten Feld untersucht. Einerseits wirkt die Differenz von Gravitations- und Auftriebskraft
\[
\vec F_g = -\frac43\pi r^3\left(\rho_{Oel}-\rho_{L}\right)\cdot \vec e_y
\]
Bewegt sich der Tropfen auf Grund dieser Kraft nach unten wirkt in entgegengesetzter Richtung die Reibungskraft $F_R$. F\"ur die Reibung nach Stokes gilt:
\[
\vec F_R = 6\pi \eta_{L}rv \cdot \vec e_y
\]
Ist 
\begin{equation}
\vec F_G + \vec F_R = \vec 0
\label{gg0}
\end{equation}
hat sich die Gleichgewichtsgeschwinndikeit $v_0$  eingestellt. Aus Bedingung \ref{gg0} kann ein Zusammenhang \"ur den Radius der Tr\"opchen $r$ ermittelt werden:
\begin{equation}
r = \sqrt{\frac{9 \eta_L v_0}{2g\left(\rho_{Oel}-\rho_{L}\right)}}\mathrm{.}
\label{rv0}
\end{equation}
\subsection{Elektrische Kr\"afte auf geladene Teilchen} 
Beindet sich das Tr\"opchen in einem elektrischen Feld und ist geladen, wirkt zus\"atzlich die elektrische Kraft
\[
\vec F_{el} = q\cdot \vec E \mathrm{.}
\]
Da das Feld durch einen Kondensator erzeugt wird gilt:
\[
\vec F_{el} = q\cdot \frac{U}{d} \cdot \vec e_y\mathrm{.}
\]
Dabei wird die Richtung der Kraft durch die Polung der Spannung am Kondensator bestimmt. Abbildung \ref{feldgg} zeigt die angreifenden K\"afte bei den verschiedenen Richtungen des Feldes mit den Gleichgewichtsgeschindikeiten $v_-$ und $v_+$. Aus dieser Grafik k\"onnen folgende Kr\"aftegleichgewichte abgelesen werden:
\begin{itemize}
\item Bei fallendem Tropfen ($v_-$):
\begin{equation}
\vec F_g+ \vec F_r - \vec F_{el}= \vec 0
\label{gg-}
\end{equation}

\item Bei steigendem Tropfen ($v_+$):
\begin{equation}
\vec F_g -\vec F_r +\vec F_{el}= \vec 0
\label{gg+}
\end{equation}
\end{itemize}
Mit diesen Gleichungen kann man Radius $r$ und Ladung $q$ der Tr\"opchen berechnen. Es gilt:

\begin{equation}
r = \sqrt{\frac{9 \eta_L (v_--v_+)}{2g\left(\rho_{Oel}-\rho_{L}\right)}}\mathrm{.}
\label{r+-}
\end{equation}

\begin{equation}
q = 3\pi\eta_L\sqrt{\frac{9 \eta_L (v_--v_+)}{4g\left(\rho_{Oel}-\rho_{L}\right)}}\cdot \frac{(v_-+v_+)d}{U}
\label{q+-}
\end{equation}
Da die Tr\"opfchen auf Grund ihrer Gr\"o\ss e das Reibunggsgesetz von Stokes nicht erf\"ullen, muss bei der Berechnung die Viskosit\"at korrigiert werden. Gem\"ass der Cunningham-Korrektur gilt:
\begin{equation}
q_{korr} = q_0 \left( 1+ \frac B{p\cdot r}\right)^{\frac32}
\end{equation}

\section{Aufbau}
Abblildung \ref{aufbau} zeigt den Aufbau der Versuchsapparatur.
\subsection{Die Milikan-Kammer}
\section{Durchf\"uhrung}
Vor Beginn der Messung muss der Versuchsaufbau kalibriert werden. Hierzu wird zun\"achst die korrekte Ausrichtung der Milikan-Apparatur mit der am Ger\"at angebrachten Libelle sichergestellt. Danach wird Kallibrations-Nadel in die Milikan-Kammer eingef\"uhrt. Hiermit kann sowohl die Tr\"opchenebene, als das Kooordinatengitter am Mikroskop scharf gestellt werden. Abchlie\ss end wird der Fokus und die Helligkeit so eingestellt, dass das innere der Kammer klar  erkennbar ist.
\subsection{Messprogramm}
Zu Beginn der Messung wird die Kondensatorsapnnung auf 
\[
U_C = \SI{300}{\volt}
\]
eingestellt. Danach wird  eine geringe Menge \"Ol in die Kammer eingebracht. Nun wird auf dem Bildschirm ein geeignetes Tr\"opchen gew\"ahlt. Dieses zeichnet sich durch eine geringe waagerechte Dirftgeschwindigkeit und eine gen\"ugende Gr\"o\ss e aus. An diesem Punkt muss sichergestellt werden, dass auf dem Tr\"opchen Ladung befindet. Dies wird durch Anlegend und anschlie\ss endes Umpolen der Kondensatorspannung festgestellt. Ist keine \"Anderung der Bewegungsrichtung erkennbar, wird versucht, mit Hilfe eines $\alpha$-Strahlers, zus\"atzliche Ladung auf das Tr\"opchen zu bringen. Ist dies nicht erfolgreich muss ein anderer Tropfen f\"ur die Messung gew\"ahlt werden.
Bei der eigentlichen Messung wird die Zeit $t$ messen, welche das Tr\"opchen f\"ur eine Strecke
\[
s = \SI{}{\milli\meter}
\]
ben\"otigt. F\"ur jeden Tropfen werden 3 Messungen durchgef\"uhrt. Die Laufzeiten werden ohne Kondensatorspannung $t_0$ und mit eingeschalteter Spannung in unterschiedlicher Polung $t_+$ und $t_-$ gemessen. Aus diesen Werten werden dann die Geschwindikeiten
\[
v = \frac{s}{t}
\]
berechnet. Dieser Vorgang wird wiederholt, bis f\"ur 15 verschiedene Tropfen die 3 Geschwindigkeiten bekannt sind.

In einem zweiten Versuchsteil wird die Kondensatorspannnung auf 
\[
U_C = \SI{}{\volt}
\]
gesenkt. F\"ur 15 weitere Tr\"opfchen werden die Laufzeiten bei eingeschalteter Spannung $t_+$ und $t_-$ bestimmt.
\section{Auswertung}
\subsection{Auswertung für $U=\SI{297}{\volt}$}
\begin{table}[H]
\centering
\begin{tabular}{cccc}
\toprule
{$t_{Null}$[s]} &{ $t_{auf}$[s]} &{ $t_{ab}$[s]} &{ R[M$\Omega$] }\\
\midrule
34.900 & 3.273 & 2.918 & 2.03\\
58.948 & 5.011 & 7.655 & 2.01\\
28.018 & 4.204 & 9.842 & 1.89\\
15.755 & 7.946 & 4.245 & 1.89\\
26.543 & 4.735 & 3.879 & 1.89\\
23.904 & 5.640 & 3.685 & 1.85\\
64.958 & 3.796 & 3.484 & 1.84\\
26.536 & 11.756 & 6.005 & 1.84\\
58.571 & 3.633 & 3.582 & 1.80\\
61.098 & 20.798 & 14.684 & 1.80\\
17.396 & 52.334 & 6.584 & 1.79\\
27.842 & 5.237 & 3.838 & 1.79\\
34.652 & 5.125 & 3.424 & 1.79\\
47.401 & 3.125 & 2.967 & 1.77\\
12.061 & 17.707 & 4.751 & 1.77\\
55.808 & 9.272 & 7.093 & 1.77\\
18.702 & 35.159 & 8.964 & 1.77\\
37.240 & 5.758 & 4.510 & 1.77\\
17.102 & 23.986 & 5.877 & 1.77\\
\bottomrule
\end{tabular}
\label{}
\caption{Messwerte mit $U=\SI{297}{\volt}$}
\end{table}
\begin{table}[H]
\hspace{-1cm}
\begin{tabular}{ccccccc}
\toprule
{$v_{auf}$[$10^{-5}$m/s]} &{ $v_{ab}$[$10^{-4}$m/s]} &{ T[$^\circ C$]} &{ $\eta_{L}[10^{-5}Nsm^{-2}]$} &{ $\eta[10^{-5}Nsm^{-2}]$} &{ r[$10^{-7}$m]} &{ q[$10^{-19}$C] }\\
\midrule
6.292 & 1.178 & 27.0 & 1.8575 & 1.671 & 7.268 & 4.671\\
8.865 & 1.357 & 28.0 & 1.8620 & 1.662 & 6.738 & 5.325\\
0.132 & 1.435 & 28.0 & 1.8620 & 1.501 & 3.374 & 2.808\\
4.253 & 0.833 & 28.0 & 1.8620 & 1.649 & 6.271 & 2.746\\
95.540 & 0.759 & 29.0 & 1.8670 & 1.695 & 8.016 & 2.484\\
9.547 & 1.303 & 29.0 & 1.8670 & 1.638 & 5.804 & 4.510\\
2.824 & 1.052 & 30.0 & 1.8720 & 1.711 & 8.645 & 4.236\\
5.393 & 0.705 & 30.0 & 1.8720 & 1.557 & 4.010 & 1.589\\
1.422 & 0.558 & 30.0 & 1.8720 & 1.660 & 6.351 & 1.559\\
8.684 & 1.109 & 30.0 & 1.8720 & 1.603 & 4.829 & 3.177\\
2.085 & 0.851 & 30.0 & 1.8720 & 1.697 & 7.895 & 3.033\\
\bottomrule
\end{tabular}
\label{}
\caption{Auswertung für $U=\SI{297}{\volt}$}
\end{table}

\subsection{Auswertung für $U=\SI{297}{\volt}$}
\begin{table}[H]
\centering
\begin{tabular}{ccc}
\toprule
{ $t_{auf}$[s]} &{ $t_{ab}$[s]} &{ R[M$\Omega$] }\\
\midrule
23.776 & 20.512 & 1.76\\
27.441 & 9.432 & 1.76\\
24.321 & 16.568 & 1.76\\
27.289 & 16.873 & 1.76\\
21.155 & 18.188 & 1.76\\
40.118 & 8.290 & 1.76\\
24.523 & 8.476 & 1.76\\
7.059 & 5.758 & 1.76\\
10.672 & 4.887 & 1.75\\
13.574 & 6.956 & 1.75\\
5.253 & 3.278 & 1.75\\
8.079 & 10.210 & 1.75\\
13.310 & 5.877 & 1.75\\
3.606 & 3.904 & 1.75\\
6.833 & 3.876 & 1.75\\
\bottomrule
\end{tabular}
\label{}
\caption{Messwerte mit $U=\SI{150}{\volt}$}
\end{table}

\begin{table}[H]
\centering
\begin{tabular}{ccccccc}
\toprule
{$v_{auf}$[$10^{-5}$m/s]} &{ $v_{ab}$[$10^{-5}$m/s]} &{ T[$^\circ C$]} &{ $\eta_{L}[10^{-5}Nsm^{-2}]$} &{ $\eta[10^{-5}Nsm^{-2}]$} &{ r[$10^{-7}$m]} &{ q[$10^{-19}$C] }\\
\midrule
2.103 & 2.438 & 30.0 & 1.872 & 1.291 & 1.802 & 0.390\\
1.822 & 5.301 & 30.0 & 1.872 & 1.643 & 5.811 & 2.829\\
2.056 & 3.018 & 30.0 & 1.872 & 1.479 & 3.056 & 0.906\\
1.832 & 2.963 & 30.0 & 1.872 & 1.504 & 3.313 & 0.951\\
2.364 & 2.749 & 30.0 & 1.872 & 1.319 & 1.934 & 0.486\\
1.246 & 6.031 & 30.0 & 1.872 & 1.673 & 6.815 & 3.484\\
2.039 & 5.899 & 30.0 & 1.872 & 1.653 & 6.121 & 3.352\\
7.083 & 8.684 & 30.0 & 1.872 & 1.552 & 3.941 & 3.902\\
4.685 & 10.231 & 31.0 & 1.877 & 1.690 & 7.346 & 7.809\\
3.684 & 7.188 & 31.0 & 1.877 & 1.648 & 5.810 & 4.355\\
9.518 & 15.253 & 31.0 & 1.877 & 1.693 & 7.470 & 13.219\\
6.189 & 4.897 & 31.0 & 1.877 & nan & nan & nan\\
3.757 & 8.508 & 31.0 & 1.877 & 1.677 & 6.800 & 5.872\\
13.866 & 12.807 & 31.0 & 1.877 & nan & nan & nan\\
7.317 & 12.900 & 31.0 & 1.877 & 1.691 & 7.370 & 10.624\\
\bottomrule
\end{tabular}
\label{}
\caption{}
\end{table}

\section{Diskussion}

\section{Quellen}
\begin{enumerate}[{[}1{]}]
\item Entnommen der Praktikumsanleitung \textit{} der TU Dortmund. \\
Download am 04.06.14 unter:\\
 \url{http://129.217.224.2/HOMEPAGE/PHYSIKER/BACHELOR/AP/SKRIPT/Milikan.pdf}
\end{enumerate}
\section{Anhang}
\begin{itemize}
\item Auszug aus dem Messheft
\end{itemize}
\end{document}
